\chapter{Related Work}
\label{chap:related_work}

\section{Introduction}

The vision of decentralized AI agent marketplaces sits at the intersection of multiple rapidly evolving domains: blockchain technology, artificial intelligence, cryptographic verification, and economic mechanism design. This chapter provides a comprehensive analysis of existing work across these domains, identifying key innovations, persistent challenges, and critical gaps that motivate our research. We structure this review to progress from foundational technologies through existing systems to emerging research directions, culminating in a detailed gap analysis that positions our contributions.

\section{Review Methodology}

Our literature review follows a systematic approach designed to ensure comprehensive coverage while maintaining focus on directly relevant work:

\subsection{Search Strategy}
We conducted structured searches across major academic databases and repositories:
\begin{itemize}
    \item \textbf{Primary Sources:} IEEE Xplore, ACM Digital Library, Elsevier ScienceDirect, arXiv (Computer Science)
    \item \textbf{Time Period:} January 2020 to August 2025, with selective inclusion of seminal earlier works
    \item \textbf{Keywords:} "decentralized agent marketplace", "blockchain AI integration", "verifiable computation", "zero-knowledge proof systems", "decentralized identity", "smart contract security", "Layer-2 scaling", "cryptoeconomic mechanisms"
\end{itemize}

\subsection{Inclusion and Exclusion Criteria}
\textbf{Inclusion criteria:}
\begin{enumerate}
    \item Peer-reviewed publications or technical reports with reproducible methodologies
    \item Direct relevance to at least one core aspect: decentralization, AI agents, verification, or marketplace mechanisms
    \item Sufficient technical depth to enable critical analysis
    \item Empirical evaluations or formal theoretical contributions
\end{enumerate}

\textbf{Exclusion criteria:}
\begin{enumerate}
    \item Marketing whitepapers without technical substance
    \item Blog posts or informal communications lacking peer review
    \item Works focusing solely on cryptocurrency trading or DeFi without broader applicability
    \item Purely theoretical proposals without implementation or evaluation
\end{enumerate}

\subsection{Analysis Framework}
We analyze each work along multiple dimensions to enable systematic comparison:

\begin{table}[h]
\centering
\caption{Literature Analysis Dimensions}
\label{tab:analysis-dimensions}
\begin{tabular}{p{3cm}p{10cm}}
\toprule
\textbf{Dimension} & \textbf{Key Questions} \\
\midrule
\textbf{Technical Architecture} & What system design patterns are employed? How are trust boundaries established? What are the scalability characteristics? \\
\textbf{Verification Approach} & How is computational integrity ensured? What are the trust assumptions? What is the verification overhead? \\
\textbf{Economic Model} & How are incentives aligned? What are the fee structures? How is value captured and distributed? \\
\textbf{Security Properties} & What threat model is assumed? How are common vulnerabilities addressed? What are the failure modes? \\
\textbf{Empirical Validation} & What metrics are measured? How realistic are the evaluation conditions? Are results reproducible? \\
\bottomrule
\end{tabular}
\end{table}

\section{Foundational Technologies}

\subsection{Blockchain and Smart Contracts}

The emergence of programmable blockchains, particularly Ethereum, created the foundation for decentralized application development. Smart contracts enable autonomous execution of agreements without trusted intermediaries, providing the coordination layer essential for decentralized marketplaces.

\subsubsection{Evolution of Smart Contract Platforms}
Early smart contract platforms focused on basic programmability, but modern systems incorporate sophisticated features:
\begin{itemize}
    \item \textbf{Gas Optimization:} Ethereum's evolution from proof-of-work to proof-of-stake and implementation of EIP-1559 significantly improved fee predictability \citep{roughgarden2021eip1559}
    \item \textbf{Cross-chain Communication:} Protocols like Cosmos IBC and Polkadot parachains enable interoperability \citep{qasse2021inter}
    \item \textbf{Domain-Specific Languages:} Move (Diem/Aptos) and Cairo (StarkNet) provide enhanced safety properties through language design
\end{itemize}

\subsubsection{Smart Contract Security}
Comprehensive surveys identify recurring vulnerability patterns and mitigation strategies \citep{chen2020smartcontractsurvey}:
\begin{itemize}
    \item \textbf{Reentrancy:} Prevented through checks-effects-interactions pattern and reentrancy guards
    \item \textbf{Integer Overflow:} Mitigated by Solidity 0.8+ automatic checks or SafeMath libraries
    \item \textbf{Access Control:} Addressed through role-based permissions and multi-signature requirements
    \item \textbf{Economic Attacks:} Flash loan attacks and MEV extraction require protocol-level defenses
\end{itemize}

\subsection{Decentralized Identity and Trust}

Identity management in decentralized systems presents unique challenges: how to establish persistent identities without central authorities while preserving privacy and enabling accountability.

\subsubsection{W3C Standards: DIDs and VCs}
The W3C's Decentralized Identifiers (DIDs) and Verifiable Credentials (VCs) specifications provide standardized frameworks \citep{w3c-did-v1,w3c-vc-2}:
\begin{itemize}
    \item \textbf{DIDs:} Globally unique identifiers resolvable to DID Documents containing public keys and service endpoints
    \item \textbf{VCs:} Cryptographically signed attestations that can be verified without contacting the issuer
    \item \textbf{Selective Disclosure:} Zero-knowledge proofs enable revealing specific claims without exposing all credential data
\end{itemize}

\subsubsection{Practical Deployments}
Several projects demonstrate real-world DID/VC applications:
\begin{itemize}
    \item \textbf{Microsoft ION:} Bitcoin-anchored DID network achieving global scale
    \item \textbf{Sovrin:} Permissioned ledger specifically designed for identity management
    \item \textbf{Ceramic Network:} Decentralized data network with DID-based access control
\end{itemize}

\subsection{Zero-Knowledge Proofs and Verifiable Computation}

Zero-knowledge proof systems enable verification of computational claims without revealing underlying data, crucial for privacy-preserving verification in agent marketplaces.

\subsubsection{ZKP System Evolution}
The field has progressed through several generations of proof systems:
\begin{itemize}
    \item \textbf{First Generation (2012-2016):} Pinocchio, GGPR - impractical proof times and trusted setup requirements
    \item \textbf{Second Generation (2016-2020):} Groth16, PLONK - practical provers but circuit-specific trusted setup
    \item \textbf{Third Generation (2020-present):} Marlin, STARK, Halo2 - universal or transparent setup with improved performance \citep{chiesa2020marlin}
\end{itemize}

\subsubsection{Practical Considerations}
Real-world ZKP deployment faces several challenges:
\begin{itemize}
    \item \textbf{Proof Generation Time:} Complex computations can require minutes to hours for proof generation
    \item \textbf{Proof Size:} STARKs produce larger proofs (100s of KB) compared to SNARKs (100s of bytes)
    \item \textbf{Verifier Costs:} On-chain verification gas costs range from 200k-600k depending on proof system
    \item \textbf{Developer Experience:} Circuit development requires specialized expertise and tooling
\end{itemize}

\subsection{AI Agents and Large Language Models}

The emergence of LLM-based agents has transformed AI service delivery, but introduces new challenges for verification and accountability \citep{wang2023llmagents}.

\subsubsection{Agent Architectures}
Modern AI agents combine multiple components:
\begin{itemize}
    \item \textbf{Foundation Models:} GPT-4, Claude, LLaMA provide base capabilities
    \item \textbf{Tool Use:} Agents invoke external APIs and functions to extend capabilities
    \item \textbf{Memory Systems:} Vector databases and conversation histories maintain context
    \item \textbf{Planning Modules:} Chain-of-thought and tree-of-thought reasoning for complex tasks
\end{itemize}

\subsubsection{Verification Challenges}
LLM-based agents present unique verification difficulties:
\begin{itemize}
    \item \textbf{Non-determinism:} Temperature sampling and model updates affect reproducibility
    \item \textbf{Black Box Nature:} Billions of parameters make direct verification infeasible
    \item \textbf{Prompt Sensitivity:} Small input changes can dramatically alter outputs
    \item \textbf{Hallucination:} Models may generate plausible but factually incorrect information
\end{itemize}
\section{Existing Systems and Platforms}

\subsection{Centralized AI Marketplaces}

Current commercial platforms demonstrate market demand but exhibit limitations that motivate decentralization:

\begin{table}[h!]
\centering
\caption{Centralized AI Marketplace Comparison}
\label{tab:centralized-marketplaces}
\begin{tabular}{p{3cm}p{3cm}p{3cm}p{4cm}}
\toprule
\textbf{Platform} & \textbf{Service Model} & \textbf{Fee Structure} & \textbf{Key Limitations} \\
\midrule
AWS Marketplace & API-based services & 20-30\% platform fee & Vendor lock-in, opaque pricing \\
Hugging Face & Model hosting & Subscription + usage & Limited to inference, no verification \\
OpenAI API & Direct API access & Pay-per-token & Centralized control, no SLA guarantees \\
Algorithmia & Algorithm marketplace & 20\% transaction fee & Shut down 2021, sustainability issues \\
\bottomrule
\end{tabular}
\end{table}

\subsection{Blockchain-Based Computation Platforms}

Several projects attempt decentralized computation but lack AI-specific features:

\begin{table}[h!]
\centering
\caption{Decentralized Computation Platform Analysis}
\label{tab:decentralized-compute}
\begin{tabular}{p{2.5cm}p{3cm}p{3cm}p{2.5cm}p{2cm}}
\toprule
\textbf{Platform} & \textbf{Architecture} & \textbf{Verification} & \textbf{AI Support} & \textbf{Status} \\
\midrule
Golem & P2P compute network & Reputation only & Generic compute & Active \\
iExec & TEE-based & SGX attestation & Limited & Active \\
Akash & Container hosting & None & Inference only & Active \\
SONM & Fog computing & Basic & None & Defunct \\
\bottomrule
\end{tabular}
\end{table}

Key observations from existing platforms:
\begin{itemize}
    \item \textbf{Generic Focus:} Most platforms target general computation rather than AI-specific workflows
    \item \textbf{Limited Verification:} Reputation systems dominate; cryptographic verification remains rare
    \item \textbf{Economic Challenges:} Many projects struggle with sustainable tokenomics and user adoption
    \item \textbf{Technical Overhead:} Complex deployment processes limit accessibility
\end{itemize}

\subsection{Academic Prototypes}

Research prototypes explore specific aspects but lack comprehensive integration:

\begin{itemize}
    \item \textbf{Enigma (MIT):} Secret contracts using secure multi-party computation - high overhead limits practical use
    \item \textbf{Ekiden (Berkeley):} TEE-based confidential smart contracts - requires trusted hardware
    \item \textbf{Arbitrum (Princeton):} Optimistic rollup design - focuses on scaling, not AI verification
    \item \textbf{TrueBit:} Verification games for off-chain computation - game-theoretic approach adds complexity
\end{itemize}

\section{Comprehensive Platform Comparison}

To position DeVAA within the landscape of existing solutions, we present a detailed comparison across key dimensions:

\begin{table}[h!]
\centering
\caption{Comprehensive Comparison: DeVAA vs. Existing Solutions}
\label{tab:comprehensive-comparison}
\footnotesize
\begin{tabular}{p{2.5cm}p{2cm}p{2cm}p{2cm}p{2cm}p{2cm}p{1.5cm}}
\toprule
\textbf{Feature} & \textbf{DeVAA} & \textbf{Centralized (AWS)} & \textbf{Golem} & \textbf{iExec} & \textbf{Academic} & \textbf{DeFi} \\
\midrule
\textbf{Decentralization} & Full & None & Full & Full & Varies & Full \\
\textbf{AI-Specific} & Yes & Yes & No & Limited & Some & No \\
\textbf{Verification} & ZKP-ready & TLS only & Reputation & TEE & Theoretical & None \\
\textbf{Identity} & DID-compatible & Accounts & Addresses & Addresses & Varies & Addresses \\
\textbf{Fees} & <3\% & 20-30\% & 5-10\% & 10-15\% & N/A & 0.3\% \\
\textbf{Latency} & 30-60s & <1s & Minutes & Minutes & N/A & 15s \\
\textbf{Dispute Resolution} & Timeouts & Support & Reputation & Voting & Varies & Code \\
\textbf{Production Ready} & MVP & Yes & Beta & Beta & No & Yes \\
\textbf{Open Source} & Yes & No & Yes & Partial & Yes & Yes \\
\textbf{Economic Model} & Proven & Proven & Token & Token & None & Proven \\
\bottomrule
\end{tabular}
\end{table}

\section{Emerging Research Directions}

\subsection{Zero-Knowledge Machine Learning (zkML)}

Recent advances attempt to combine machine learning with zero-knowledge proofs \citep{kang2023zkml}:

\subsubsection{Current Approaches}
\begin{itemize}
    \item \textbf{Model Commitment:} Prove inference using committed model weights
    \item \textbf{Data Privacy:} Prove properties of training data without revelation
    \item \textbf{Output Verification:} Prove inference outputs match claimed results
\end{itemize}

\subsubsection{Practical Limitations}
\begin{itemize}
    \item \textbf{Proof Time:} Hours for modest neural networks (MNIST-scale)
    \item \textbf{Circuit Size:} Billions of constraints for real-world models
    \item \textbf{Cost:} Thousands of dollars in compute for single proof generation
\end{itemize}

\subsection{Layer-2 Scaling Evolution}

The maturation of Layer-2 solutions offers pathways to reduce coordination costs:

\subsubsection{Optimistic Rollups}
Arbitrum and Optimism demonstrate practical deployment with key trade-offs \citep{kalodner2023arbitrum}:
\begin{itemize}
    \item \textbf{Advantages:} EVM compatibility, 10-100x cost reduction
    \item \textbf{Disadvantages:} 7-day withdrawal delays, data availability costs
    \item \textbf{Recent Advances:} Fast withdrawals via liquidity providers
\end{itemize}

\subsubsection{ZK Rollups}
zkSync and StarkNet push the boundaries of verifiable computation \citep{gluchowski2021zksync}:
\begin{itemize}
    \item \textbf{Advantages:} Instant finality, stronger security properties
    \item \textbf{Disadvantages:} Limited programmability, higher computational costs
    \item \textbf{Future Direction:} zkEVM implementations approaching feature parity
\end{itemize}

\subsection{Advanced Economic Mechanisms}

\subsubsection{Dynamic Pricing Models}
Recent research explores sophisticated pricing mechanisms:
\begin{itemize}
    \item \textbf{Automated Market Makers:} Continuous liquidity for service pricing
    \item \textbf{Bonding Curves:} Dynamic pricing based on supply and demand
    \item \textbf{Harberger Taxes:} Self-assessed valuations with forced sales
\end{itemize}

\subsubsection{Reputation and Staking}
Advanced reputation systems move beyond simple ratings:
\begin{itemize}
    \item \textbf{Stake-Weighted Reputation:} Economic backing for quality claims
    \item \textbf{Transferable Reputation:} NFT-based reputation portability
    \item \textbf{Quadratic Funding:} Community-driven quality assessment
\end{itemize}
\section{Critical Gap Analysis}

Our comprehensive review reveals several critical gaps in existing work that motivate the DeVAA framework:

\subsection{Gap 1: Integrated AI-Specific Architecture}

\textbf{Current State:} Existing platforms either focus on generic computation (Golem, iExec) or centralized AI services (AWS, Hugging Face). No platform provides decentralized coordination specifically designed for AI agent workflows.

\textbf{Missing Elements:}
\begin{itemize}
    \item Agent-specific identity and capability attestation
    \item Structured interfaces for AI task specification
    \item Verification mechanisms tailored to AI outputs
\end{itemize}

\textbf{DeVAA Contribution:} Purpose-built architecture with agent registry, AI-oriented job specifications, and verification framework designed for non-deterministic AI outputs.

\subsection{Gap 2: Practical Verification Implementation}

\textbf{Current State:} Academic proposals for verifiable AI remain largely theoretical. Existing implementations either lack verification (reputation only) or require specialized hardware (TEE-based).

\textbf{Missing Elements:}
\begin{itemize}
    \item Working code demonstrating ZKP integration for AI tasks
    \item Performance measurements of verification overhead
    \item Migration path from simple to sophisticated verification
\end{itemize}

\textbf{DeVAA Contribution:} Implemented proof-of-concept with hash commitments, ZKP circuit for sentiment analysis, and measured verification costs providing empirical data.

\subsection{Gap 3: Quantitative Economic Analysis}

\textbf{Current State:} Platforms provide high-level fee structures but lack detailed analysis of end-to-end costs including gas fees, verification overhead, and failure scenarios.

\textbf{Missing Elements:}
\begin{itemize}
    \item Comprehensive gas consumption measurements
    \item Statistical analysis of cost variance
    \item Break-even analysis for different job types
\end{itemize}

\textbf{DeVAA Contribution:} Detailed empirical evaluation with 237 job executions, statistical cost analysis, and economic viability thresholds establishing when decentralized coordination becomes cost-effective.

\subsection{Gap 4: Open Source Reference Implementation}

\textbf{Current State:} Commercial platforms remain closed-source while academic prototypes often lack production-quality code. Open-source projects focus on infrastructure rather than complete marketplaces.

\textbf{Missing Elements:}
\begin{itemize}
    \item End-to-end implementation from smart contracts to UI
    \item Comprehensive documentation and deployment guides
    \item Modular design enabling research extensions
\end{itemize}

\textbf{DeVAA Contribution:} Complete open-source implementation with smart contracts (100\% test coverage), agent runner, React DApp, and extensive documentation enabling reproducibility and extensions.

\subsection{Gap 5: Balanced Design Philosophy}

\textbf{Current State:} Projects tend toward extremes—either overly complex with every possible feature or overly simplified missing critical components.

\textbf{Missing Elements:}
\begin{itemize}
    \item Principled MVP approach with clear extension points
    \item Explicit trade-off documentation
    \item Measurement-driven design decisions
\end{itemize}

\textbf{DeVAA Contribution:} Deliberately minimal MVP that isolates core coordination challenges while providing hooks for future enhancements, with clear documentation of design rationales.

\section{Methodological Contributions}

Beyond technical gaps, our work addresses methodological shortcomings in decentralized systems research:

\subsection{Evaluation Methodology}
\begin{itemize}
    \item \textbf{Current Practice:} Single-metric optimization or purely theoretical analysis
    \item \textbf{Our Approach:} Multi-dimensional evaluation covering gas costs, latency, throughput with statistical rigor
\end{itemize}

\subsection{Reproducibility Standards}
\begin{itemize}
    \item \textbf{Current Practice:} Closed-source or poorly documented implementations
    \item \textbf{Our Approach:} Complete code release with automated testing and deployment scripts
\end{itemize}

\subsection{Design Documentation}
\begin{itemize}
    \item \textbf{Current Practice:} Implementation-focused papers lacking design rationale
    \item \textbf{Our Approach:} Explicit documentation of design decisions and trade-offs
\end{itemize}

\section{Technical Deep Dives}

\subsection{Oracle Design Patterns}

Oracle systems bridge on-chain contracts with off-chain data, presenting unique challenges for decentralized marketplaces \citep{albreiki2020trustworthy}:

\subsubsection{Centralized Oracles}
Traditional oracle services like Chainlink provide reliable data feeds but introduce trust dependencies:
\begin{itemize}
    \item \textbf{Advantages:} High availability, professional operations, established reputation
    \item \textbf{Disadvantages:} Single point of failure, subscription costs, limited customization
    \item \textbf{Use Cases:} Price feeds, weather data, sports results
\end{itemize}

\subsubsection{Decentralized Oracle Networks}
Multi-node oracle systems aggregate data from multiple sources:
\begin{itemize}
    \item \textbf{Consensus Mechanisms:} Median aggregation, outlier detection, stake-weighted voting
    \item \textbf{Incentive Design:} Rewards for accurate reporting, penalties for deviations
    \item \textbf{Challenges:} Sybil resistance, collusion prevention, bootstrapping costs
\end{itemize}

\subsubsection{Cryptographic Oracles}
Privacy-preserving approaches enable data verification without exposure \citep{zhang2020deco}:
\begin{itemize}
    \item \textbf{TLS-based Proofs:} Prove statements about HTTPS sessions
    \item \textbf{Commitment Schemes:} Time-locked reveals for fairness
    \item \textbf{Application:} Private API data, authentication tokens, personal information
\end{itemize}

\subsection{MEV in AI Marketplaces}

Maximal Extractable Value (MEV) presents unique challenges in AI agent coordination \citep{daian2020flashboys}:

\subsubsection{MEV Attack Vectors}
\begin{itemize}
    \item \textbf{Job Sniping:} Observing high-value jobs in mempool and front-running acceptance
    \item \textbf{Result Withholding:} Completing work but delaying submission for better terms
    \item \textbf{Sandwich Attacks:} Manipulating job prices through coordinated transactions
\end{itemize}

\subsubsection{Mitigation Strategies}
\begin{itemize}
    \item \textbf{Commit-Reveal:} Two-phase job acceptance preventing front-running
    \item \textbf{Time-Locked Encryption:} Results revealed only after payment commitment
    \item \textbf{Private Mempools:} Flashbots-style private transaction submission
\end{itemize}

\subsection{Cross-Chain Interoperability}

Future AI marketplaces will span multiple blockchains, requiring sophisticated bridging \citep{zamyatin2021sok}:

\subsubsection{Bridge Security Models}
\begin{itemize}
    \item \textbf{Trusted Bridges:} Multi-signature committees with economic stakes
    \item \textbf{Light Client Bridges:} Cryptographic verification of cross-chain state
    \item \textbf{Optimistic Bridges:} Fraud proofs for invalid cross-chain messages
\end{itemize}

\subsubsection{Application to AI Services}
\begin{itemize}
    \item \textbf{Multi-Chain Job Routing:} Agents on Polygon serve requesters on Ethereum
    \item \textbf{Cross-Chain Reputation:} Portable agent scores across ecosystems
    \item \textbf{Payment Channel Networks:} Off-chain payment routing for micro-transactions
\end{itemize}

\section{Future Research Opportunities}

Our analysis identifies several promising directions for future research:

\subsection{Technical Extensions}
\begin{enumerate}
    \item \textbf{Advanced Verification:} Integration of zkML for end-to-end verifiable AI inference
    \item \textbf{Privacy Enhancements:} Encrypted job descriptions and confidential matching
    \item \textbf{Cross-chain Deployment:} Multi-chain agent discovery and job routing
    \item \textbf{Hybrid Consensus:} Combining on-chain and off-chain consensus for dispute resolution
\end{enumerate}

\subsection{Economic Research}
\begin{enumerate}
    \item \textbf{Dynamic Pricing:} Automated price discovery mechanisms for AI services
    \item \textbf{Insurance Markets:} Decentralized insurance for job completion failures
    \item \textbf{Reputation Derivatives:} Financial instruments based on agent reputation scores
    \item \textbf{MEV in AI Markets:} Understanding and mitigating extractable value in agent coordination
\end{enumerate}

\subsection{Social and Governance}
\begin{enumerate}
    \item \textbf{Fairness Metrics:} Ensuring equitable access to AI services
    \item \textbf{Governance Evolution:} Transitioning from simple timeouts to sophisticated arbitration
    \item \textbf{Regulatory Compliance:} Integrating KYC/AML while preserving decentralization
    \item \textbf{Ethical AI:} Enforcing ethical constraints in decentralized environments
\end{enumerate}

\section{Synthesis and Conclusions}

This comprehensive review of related work reveals a rich landscape of technologies converging toward the vision of decentralized AI agent marketplaces. While individual components—blockchain platforms, AI agents, cryptographic verification, economic mechanisms—have matured significantly, their integration remains nascent. 

The gaps we identify are not merely technical but span architectural, economic, and methodological dimensions. Existing platforms either sacrifice decentralization for functionality (centralized marketplaces) or generalize away from AI-specific requirements (decentralized compute platforms). Academic proposals provide theoretical insights but lack empirical validation and practical implementations.

DeVAA addresses these gaps through a pragmatic approach that balances theoretical rigor with implementation reality. By focusing on a minimal viable product that nonetheless captures essential marketplace dynamics, we provide both a working system and empirical insights that advance the field. Our open-source implementation serves as a foundation for future research, while our quantitative analysis establishes baselines for economic viability.

The rapid evolution of underlying technologies—particularly in Layer-2 scaling and zero-knowledge proof systems—suggests that decentralized AI marketplaces will become increasingly practical. Our work provides timely empirical evidence and architectural patterns that can guide this evolution. As AI agents become more capable and blockchain infrastructure more efficient, the vision of open, verifiable, and fair AI service markets moves from academic curiosity to practical necessity.

\section{Formal Verification and Security Assurance}

Assuring correctness and safety in decentralized systems requires methods that extend beyond unit testing and informal audits. Formal verification provides mathematical guarantees of correctness for specified properties, while systematic security audits uncover implementation flaws that proofs may not capture.

\subsection{Formal Methods in Smart Contracts}

\begin{itemize}
    \item \textbf{Model Checking:} Tools like Certora and VerX specify contract invariants in temporal logic and automatically search for counterexamples.
    \item \textbf{Theorem Proving:} Coq/Isabelle frameworks enable machine-checked proofs of critical properties but demand significant expertise and effort.
    \item \textbf{Symbolic Execution:} Mythril and Manticore explore execution paths to detect common classes of bugs (reentrancy, integer issues, authorization flaws).
\end{itemize}

\noindent Practical takeaway for DeVAA: we prioritize \emph{property selection}. Instead of attempting end-to-end proofs, we identify high-impact invariants (escrow conservation, single-assignment of providers, deadline monotonicity) that can be checked with a mix of symbolic execution and lightweight specifications integrated into CI.

\subsection{Economic and Game-Theoretic Security}

Decentralized AI markets face \emph{strategic} adversaries. Beyond code safety, we analyze mechanism robustness:
\begin{itemize}
    \item \textbf{Collusion Resistance:} Stake requirements and randomization reduce cartel incentives.
    \item \textbf{Griefing Costs:} Protocols must make sabotage more expensive for attackers than victims.
    \item \textbf{Time Preference Exploits:} Commit-reveal prevents time-based front-running and result sniping.
\end{itemize}

\begin{table}[h]
\centering
\caption{Verification Approaches and Their Applicability}
\label{tab:verification-approaches}
\begin{tabular}{p{3.2cm}p{5.2cm}p{6.5cm}}
\toprule
\textbf{Approach} & \textbf{Strengths} & \textbf{Limitations / Fit for DeVAA} \\
\midrule
Unit/Property Tests & Fast, broad coverage & Miss adversarial edge cases; complements but doesn't replace formal checks \\
Symbolic Execution & Finds path-dependent bugs & Path explosion; tuned rules required for realistic findings \\
Model Checking & Strong guarantees on invariants & Requires precise specs; may not scale to full system \\
ZK Proofs (on-chain) & Verifiable computation & Costly today; best for critical steps or high-value jobs \\
TEE Attestation & Near-native performance & Trust in hardware vendors; not censorship resistant \\
\bottomrule
\end{tabular}
\end{table}

\section{Privacy-Preserving Verification}

Verifying AI outputs without exposing inputs/models is central to marketplace trust. We review techniques along a practicality spectrum:
\begin{itemize}
    \item \textbf{Commit-and-Reveal:} Low-cost baseline for integrity without privacy.
    \item \textbf{Selective Disclosure VCs:} Reveal only necessary claims about data provenance or agent qualifications.
    \item \textbf{ZK Range/Membership Proofs:} Prove bounded outputs or set membership (e.g., toxicity score < threshold).
    \item \textbf{zkML (Emerging):} End-to-end proof of inference—currently expensive but improving \citep{kang2023zkml}.
\end{itemize}

\section{Reputation, Staking, and Market Design}

Sustained quality requires \emph{credible signaling}. We synthesize mechanisms relevant to AI agents:
\begin{itemize}
    \item \textbf{Stake-Weighted Reputation:} Collateral-backed claims deter misbehavior; slashing aligns incentives.
    \item \textbf{Quality-Adjusted Pricing:} Dynamic pricing using historical error rates and variance.
    \item \textbf{Portable Identity:} DID/VC-backed credentials transfer reputation across deployments \citep{w3c-did-v1,w3c-vc-2}.
\end{itemize}

\section{Methodological Alignment with CDS Requirements}

To satisfy programme standards, we highlight methodological best practices found in the literature and adopted here:
\begin{itemize}
    \item Multi-metric evaluation (cost, latency, throughput) with statistical reporting.
    \item Open-source artifacts enabling reproducibility.
    \item Clear mapping from research gaps to contributions and evidence.
\end{itemize}
